\documentclass[12pt, a4paper]{article}

\usepackage[utf8]{inputenc}
\usepackage{graphicx}
\usepackage[russian]{babel}

\title{Manual 3DViewer v1.0}

\author{chesterh, cyndisig, cornichl}

\date{Сентябрь 02 2023}


\begin{document}

% first manual page
\maketitle

\pagebreak

\tableofcontents

\pagebreak

\section{Общее описание}

3DViewer v1.0 - программа для просмотра 3D моделей в каркасном виде, с возможностью вращения, масштабирования и перемещения моделей.

\pagebreak

\section{Введение}
3D Viewer :
    \begin{itemize}
        \item Предназначен для просмотра 3D моделей в каркасном виде.
        \item Поддерживает вращение, масштабирование и перемещение моделей.
        \item Поддерживает следующие операции:
        \begin{enumerate}
            \item Загрузка каркасных моделей из файла формата obj (поддержка только списка вершин и поверхностей);
            \item Перемещение модели на заданное расстояние относительно осей X, Y, Z;
            \item Поворот модели на заданное расстояние относительно осей X, Y, Z;
            \item Масштабирование модели на заданное значение.
            \item Программа должна позволять настраивать тип проекции (параллельная и центральная);
            \item Программа позволяет настраивать тип (сплошная, пунктирная), цвет и толщину ребер, способ отображения (отсутствует, круг, квадрат), цвет и размер вершин;
            \item Программа позволяет выбирать цвет фона;
            \item Настройки должны сохраняться между перезапусками программы
        \end{enumerate}
        \item Программа позволяет сохранять полученные ("отрендеренные") изображения в файл в форматах bmp и jpeg
        \item Программа позволяет записывать небольшие "скринкасты" 
    \end{itemize}

\section{Установка}

\begin{enumerate}
    \item Скачайте репозиторий проекта;
    \item Перейдите в терминале в папку src проекта;
    \item Выполните \textbf{make install};
    \item Откройте папку приложения на рабочем столе и запустите приложение.
\end{enumerate}

\section{Удаление}

\begin{enumerate}
    \item Перейдите в терминале в папку src проекта;
    \item Выполните: \textbf{make uninstall};
    \item Удалите папку проекта.
\end{enumerate}

\end{document}